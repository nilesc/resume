%% start of file `template.tex'.
%% Copyright 2006-2013 Xavier Danaux (xdanaux@gmail.com).
%
% This work may be distributed and/or modified under the
% conditions of the LaTeX Project Public License version 1.3c,
% available at http://www.latex-project.org/lppl/.


\documentclass[11pt,a4paper,sans]{moderncv}        % possible options include font size ('10pt', '11pt' and '12pt'), paper size ('a4paper', 'letterpaper', 'a5paper', 'legalpaper', 'executivepaper' and 'landscape') and font family ('sans' and 'roman')

% moderncv themes
\moderncvstyle{casual}                             % style options are 'casual' (default), 'classic', 'oldstyle' and 'banking'
\moderncvcolor{blue}                               % color options 'blue' (default), 'orange', 'green', 'red', 'purple', 'grey' and 'black'
%\renewcommand{\familydefault}{\sfdefault}         % to set the default font; use '\sfdefault' for the default sans serif font, '\rmdefault' for the default roman one, or any tex font name
%\nopagenumbers{}                                  % uncomment to suppress automatic page numbering for CVs longer than one page

% character encoding
\usepackage[utf8]{inputenc}                       % if you are not using xelatex ou lualatex, replace by the encoding you are using
%\usepackage{CJKutf8}                              % if you need to use CJK to typeset your resume in Chinese, Japanese or Korean

% adjust the page margins
\usepackage[scale=0.75]{geometry}
%\setlength{\hintscolumnwidth}{3cm}                % if you want to change the width of the column with the dates
%\setlength{\makecvtitlenamewidth}{10cm}           % for the 'classic' style, if you want to force the width allocated to your name and avoid line breaks. be careful though, the length is normally calculated to avoid any overlap with your personal info; use this at your own typographical risks...

% personal data
\name{Niles}{Christensen}
% \title{Resumé title}                               % optional, remove / comment the line if not wanted
\address{5220 Fifth Ave.}{Pittsburgh}{PA, 15232}
\phone[mobile]{+1~(650)~714~3983}                   % optional, remove / comment the line if not wanted
\email{nilesnchristensen@gmail.com}                               % optional, remove / comment the line if not wanted

% to show numerical labels in the bibliography (default is to show no labels); only useful if you make citations in your resume
%\makeatletter
%\renewcommand*{\bibliographyitemlabel}{\@biblabel{\arabic{enumiv}}}
%\makeatother
%\renewcommand*{\bibliographyitemlabel}{[\arabic{enumiv}]}% CONSIDER REPLACING THE ABOVE BY THIS

% bibliography with mutiple entries
%\usepackage{multibib}
%\newcites{book,misc}{{Books},{Others}}
%----------------------------------------------------------------------------------
%            content
%----------------------------------------------------------------------------------
\begin{document}
%-----       letter       ---------------------------------------------------------
% recipient data
\recipient{Waymo}{100 Mayfield Ave.\\Mountain View\\CA}
\date{October 21, 2020}
\opening{To whom it may concern,}
\closing{Sincerely,}
\makelettertitle
My name is Niles Christensen, and I’m writing to seek employment with Waymo in the role of Machine Learning Engineer. Waymo’s mission will revolutionize large parts of the
human experience, doing everything from preventing human death and suffering due to automative accidents, to freeing humans up to do things more productive and engaging than driving cars.
Waymo’s need for skilled computer scientists with machine learning backgrounds makes me an ideal candidate to help further this work.

I’m currently a graduate student at Carnegie Mellon University pursuing a Master's in Machine Learning. I previously completed my undergraduate studies at Columbia University,
where I graduated magna cum laude with a BA in Computer Science on the Intelligent Systems track. I have experience both in research settings and in industry, having worked
as an intern at the Google X project, Loon. I have worked on several research projects; of particular relevance, I'm currently working
on building a visual odometry system that uses geometrically-aware recurrent neural networks to predict the motion of a car from lidar readings.

While at Loon, I improved facets of their altitude- and wind-based steering algorithms. Though many of the details are secret, this involved processing huge amounts of data and reasoning about complex and unintuitive systems,
as well as the ability to experiment and iterate quickly. Additionally, it required the flexibility to work in the fast-paced and constantly-changing environment of Google X.
All of this makes me an ideal candidate to join Waymo and would enable me to get up to speed and start contributing as quickly as possible.

Thank you for your time and consideration. I look forward to hearing back from you and hope to be able to work together in the future.

\makeletterclosing

\end{document}


%% end of file `template.tex'.
